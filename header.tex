\usepackage[utf8]{inputenc}

\usetheme{UNIPD}
% First one is logo in title slide (we recommend use a horizontal image), and second one is the logo used in the remaining slides (we recommend use a square image)
\setLogos{lib/logos/dei_white.png}{lib/logos/dei_white_vertical.png} 

\usepackage{listings}
\usepackage[listings]{tcolorbox}

\usepackage{pgfplots}
\pgfplotsset{compat=1.3}
\tikzset{pointille/.style={dash pattern = on 2pt off 2pt on 6pt off 2pt}}
\tikzset{points/.style={dash pattern = on 1pt off 1pt}}
\tikzset{tirets/.style={dash pattern = on 5pt off 5pt}}

%\renewcommand{\only}{}

\usetikzlibrary{positioning}
\tikzset{>=stealth}
\newcommand{\tikzmark}[3][]{\tikz[remember picture,baseline] \node [anchor=base,#1](#2){$#3$};}
\usetikzlibrary{automata}

\usepackage{booktabs} 
\usepackage{bm}
\usepackage{subcaption}
\usepackage{mathrsfs}

\usepackage{palatino}

\usepackage{scrextend}
\usepackage{wrapfig}

\usepackage{quiver}
%--------------------------------------- Drawings
\usepackage{tikz}
\usetikzlibrary{calc,decorations.pathmorphing,patterns}
\pgfdeclaredecoration{penciline}{initial}{
    \state{initial}[width=+\pgfdecoratedinputsegmentremainingdistance,
    auto corner on length=1mm,]{
        \pgfpathcurveto%
        {% From
            \pgfqpoint{\pgfdecoratedinputsegmentremainingdistance}
                      {\pgfdecorationsegmentamplitude}
        }
        {%  Control 1
        \pgfmathrand
        \pgfpointadd{\pgfqpoint{\pgfdecoratedinputsegmentremainingdistance}{0pt}}
                    {\pgfqpoint{-\pgfdecorationsegmentaspect
                     \pgfdecoratedinputsegmentremainingdistance}%
                               {\pgfmathresult\pgfdecorationsegmentamplitude}
                    }
        }
        {%TO 
        \pgfpointadd{\pgfpointdecoratedinputsegmentlast}{\pgfpoint{1pt}{1pt}}
        }
    }
    \state{final}{}
}

%-------------------------------------------------------

\graphicspath{ {./imgs/}}

\usepackage[backend=bibtex, style=ieee]{biblatex}
\addbibresource{ref.bib}
\nocite{*}
\renewcommand*{\bibfont}{\tiny}

\newcommand\numberthis{\addtocounter{equation}{1}\tag{\theequation}}

%-------------------------------------theorems--------------
\newtheorem*{conjecture}{Conjecture}
\newtheorem{ex}{Example}
\newtheorem{exercise}{Exercise}
\newtheorem{lem}{Lemma}
\newtheorem{proposition}{Proposition}
\newtheorem{thm}{Theorem}
\newtheorem{cor}{Corollary}
\newtheorem*{remark}{Remark}
\newtheorem*{defi}{Definition}
\theoremstyle{definition}
\newtheorem*{assumption}{Assumption}
\newtheorem{prob}{Switching control problem}

\setbeamertemplate{theorems}[numbered]
\setbeamertemplate{caption}[numbered]

%-------------------------------------------------------------%
%----------------------- Primary Definitions -----------------%

% This command set the default Color, is also possible to choose a custom color
\setPrimaryColor{UNIPDred} 

\definecolor{simpleBlack}{HTML}{202020}

\definecolor{c1}{HTML}{7FB069}
\definecolor{c2}{HTML}{A3E7FC}
\definecolor{c3}{HTML}{FB8B24}
\definecolor{c4}{HTML}{335C67}
\definecolor{c5}{HTML}{1B9AAA}
\definecolor{c7}{HTML}{4B4E6D}
\definecolor{c6}{HTML}{F38D68}
\definecolor{c8}{HTML}{F5D547}
\definecolor{c9}{RGB}{231,237,57}
\definecolor{c10}{RGB}{237,57,71}
\definecolor{c11}{RGB}{119,237,57}
\definecolor{c12}{RGB}{0,63,225}
\definecolor{c13}{RGB}{237,57,223}


\newcommand{\ket}[1]{\left| #1 \right>}
\newcommand{\bra}[1]{\left< #1 \right|}
\newcommand{\norm}[1]{\left|\left| #1 \right|\right|}
\newcommand{\inner}[2]{\left< #1,#2 \right>}
\newcommand{\Outer}[2]{ \left|#1\right>\left<#2\right| }
\newcommand{\expect}[1]{\left< #1 \right>}
\newcommand{\blue}{\color{blue}}
\newcommand{\red}[1]{{\color{red} #1}}
\newcommand{\virg}[1]{\lq\lq#1\ignorespacesafterend\rq\rq}

\newcommand{\tr}{{\rm tr}}
\newcommand{\diag}{{\rm diag}}
\newcommand{\Span}{{\rm span}}
\newcommand{\cl}{{\rm cl}}
\DeclareMathOperator{\rank}{rank}
\newcommand{\spec}{{\rm spec}}
\newcommand{\idem}{ {\rm idem}}
\newcommand{\res}{{\rm res}}
\newcommand{\supp}{{\rm supp}}
\newcommand{\lind}{\mathcal{L}}
\newcommand{\one}{\bm 1}
\newcommand{\zero}{\bm 0}
\newcommand{\alg}{{\rm alg}}
\newcommand{\img}{{\rm Im}}

\renewcommand{\H}{\mathcal{H}}
\renewcommand{\P}{\mathbb{P}}
\newcommand{\BH}{\mathcal{B}(\H)}
\newcommand{\hH}{\mathfrak{h}(\H)}
\renewcommand{\DH}{\mathfrak{D}(\H)}
\newcommand{\Piso}{\mathcal{P}}

\newcommand{\Prob}{{\mathbb{P}}} % Probability measure in the Algebraic framework
\newcommand{\E}{{\mathbb{E}}} % Expectation in the algebraic framework
\newcommand{\str}[1]{{[{\bm #1}]}}
\newcommand{\chr}[1]{#1}

\newcommand{\R}{\mathcal{R}}
\newcommand{\NR}{\mathcal{N^R}}
\newcommand{\Oo}{\mathcal{O}}
\newcommand{\NO}{\mathcal{N^O}}
\newcommand{\p}{\bar{\bm p}}

\newcommand{\pO}{\p\wedge\Oo}
\newcommand{\pNR}{\p\wedge\NR}
\newcommand{\pNO}{\p^{-1}\wedge\NO}
\newcommand{\pR}{\p^{-1}\wedge\R}
\newcommand{\X}{\mathbb{R}^n}
\newcommand{\A}{\mathscr{A}}
\newcommand{\pA}{\p\wedge\A}

%%% ------------------------------ code snippets

\definecolor{dot1}{HTML}{FF5F55}
\definecolor{dot2}{HTML}{FFBE2A}
\definecolor{dot3}{HTML}{22CA3D}

\definecolor{background}{RGB}{39, 40, 34}
\definecolor{string}{RGB}{230, 219, 116}
\definecolor{comment}{HTML}{489038}
\definecolor{normal}{RGB}{248, 248, 242}
\definecolor{identifier}{RGB}{166, 226, 46}

\lstdefinestyle{mystyle}{
    language=Matlab,                			% choose the language of the code
  numbers=left,                   		% where to put the line-numbers
  stepnumber=1,                   		% the step between two line-numbers. 
  lineskip=-5pt,
  numbersep=5pt,                  		% how far the line-numbers are from the code
  numberstyle=\tiny\color{string}\ttfamily,
  %backgroundcolor=\color{background},  		% choose the background color. You must add \usepackage{color}
  showspaces=false,               		% show spaces adding particular underscores
  showstringspaces=false,         		% underline spaces within strings
  showtabs=false,                 		% show tabs within strings adding particular underscores
  tabsize=4,                      		% sets default tabsize to 2 spaces
  captionpos=b,                   		% sets the caption-position to bottom
  breaklines=true,                		% sets automatic line breaking
  breakatwhitespace=true,         		% sets if automatic breaks should only happen at whitespace
  %title=\lstname,                 		% show the filename of files included with \lstinputlisting;
  basicstyle=\small\color{normal}\ttfamily,					% sets font style for the code
  keywordstyle=\small\color{magenta}\ttfamily,	% sets color for keywords
  stringstyle=\small\color{string}\ttfamily,		% sets color for strings
  commentstyle=\small\color{comment}\ttfamily,	% sets color for comments
  emph={format_string, eff_ana_bf, permute, eff_ana_btr},
  emphstyle=\color{identifier}\ttfamily, 
  aboveskip=0pt
}

\lstset{style=mystyle}

% First one is logo in title slide (we recommend use a horizontal image), and second one is the logo used in the remaining slides (we recommend use a square image)
\setLogos{lib/logos/dei_white.png}{lib/logos/dei_white_vertical.png} 

\newtcbox{\codebox}[1]{colback=white!5!simpleBlack,
colframe=simpleBlack,fonttitle=\bfseries, top= 0mm, drop fuzzy shadow, enhanced, hbox,
title={\hspace{-0.3cm}{\color{dot1} $\bullet$} {\color{dot2} $\bullet$} {\color{dot3} $\bullet\quad$} #1}}


\newcommand{\codesection}[2]{
\codebox{#1}{\vspace{-0.3cm}{\lstinputlisting[language=Matlab]{#2}}}}

\tcbuselibrary{skins}

\newtcblisting{code}[1]{listing only, 
colback=white!5!simpleBlack,
colframe=simpleBlack,
fonttitle=\bfseries, 
listing options = {style = mystyle},
top= 2mm,
drop fuzzy shadow, enhanced, hbox,
title={\hspace{-0.3cm}
{\color{dot1} $\bullet$} 
{\color{dot2} $\bullet$}
{\color{dot3} $\bullet\quad$} #1}}
